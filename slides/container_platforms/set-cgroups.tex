\begin{frame}[fragile]{Setting cgroup Limits}

We are going to set the maximum memory limits to 8gb. \textbf{Note:} You can always use manually cgroups interface for setting limits.

\begin{block}{Docker}
\begin{lstlisting}[language=bash,keywordstyle=\bf,stringstyle=\it,basicstyle=\tiny]
$ docker run -it   --memory=8g ubuntu /bin/bash
\end{lstlisting}
\end{block}

\begin{block}{runc}
\begin{lstlisting}[language=bash,keywordstyle=\bf,stringstyle=\it,basicstyle=\tiny]
$ runc update --memory 8000000000 c1
\end{lstlisting}
\end{block}



\begin{block}{rkt}
\begin{lstlisting}[language=bash,keywordstyle=\bf,stringstyle=\it,basicstyle=\tiny]

Use systemd config:
[Service]
Slice=machine.slice
MemoryLimit=8G
ExecStart=/usr/bin/rkt run --interactive docker://ubuntu --insecure-options=image --volume data,kind=host,source=/data --mount volume=data,target=/data2
ExecStopPost=/usr/bin/rkt gc --mark-only
KillMode=mixed
Restart=always
\end{lstlisting}
\end{block}

\begin{block}{LXC}
\begin{lstlisting}[language=bash,keywordstyle=\bf,stringstyle=\it,basicstyle=\tiny]
$ lxc config set c1 limits.memory 8g
\end{lstlisting}
\end{block}

\end{frame}


\begin{frame}[fragile]{Notes on cgroup interaction}
So how does each interface with cgroups?
\begin{block}{Docker and runc}
\begin{itemize}
\tiny
\item Docker runs atop runc.
\item Relies on libcontainer from Docker.
\item libcontainer provides all functionality for creating containers.
\item libcontainer allows interfacing with cgroups using the native interface or systemd.
\end{itemize}
\end{block}

\begin{block}{rkt}
\tiny
\begin{itemize}
\item Takes advantage of the systemd interface for cgroups. 
\item Relies on setting resource limits in a systemd service definition rather than implementing their own system. 
\item Only supports: cpu, cpuset and memory. 
\end{itemize}
\end{block}

\begin{block}{LXC}
\begin{itemize}
\tiny
\item Uses liblxc for interacting with cgroups.
\item Filesystem limits can only be set when using zfs or btrfs. 
\end{itemize}
\end{block}

\end{frame}